\documentclass{article}
\usepackage{color}
\usepackage{accsupp}
\usepackage{hyperref}
\usepackage[ngerman]{babel}
\usepackage[latin1]{inputenc} 
\definecolor{bisque}{rgb}{0.8,0.8,0.92}
\definecolor{green}{rgb}{0.1,0.6,0.2}
\definecolor{brown}{rgb}{0.8,0.4,0.3}

\usepackage{listings}
\lstdefinestyle{basic}
{
  basicstyle=\ttfamily, % tiny,scriptsize,footnotesize,small,normalsize,large,Large,LARGE,huge,Huge
  stringstyle=\color{green}\ttfamily,
  frame=tb, % single none
  captionpos=b,
  columns=fullflexible, % to prevent spaces between characters on copying
  resetmargins=true,
  showspaces=false, % display special space character
  showstringspaces=false, % display special space character in strings
  keepspaces=true, % to copy spaces (not for leading spaces)
  backgroundcolor=\color{bisque}, % background color
  numbers=left, % linenumbers
  numberstyle=\noncopynumber, % line numbers are not copyable
  keywordstyle=\color{blue}, % keyword color
  breaklines=true,
  tabsize=4,
  commentstyle=\color{brown}
}
\newcommand{\noncopynumber}[1]{%
    \BeginAccSupp{method=escape,ActualText={}}%
    #1%
    \EndAccSupp{}%
}




\title{i6engine Tutorial 3 \\ Loading Levels}

\begin{document}

\section{Requirements}

Tutorial 2 - Adding GameObjects

\section{Introduction}

Currently, you should be able to write templates for your predefined objects and create objects out of them. Although this is enough for creating a whole world, it lacks of flexibility and would result in a lot of code. To face this problem, the i6engine provides a simple function: \textit{loadLevel()} that loads a level from a given xml file. In This tutorial, we show you how to use this function.

\section{The XML}

Take a look at listing \ref{code:xml}. It demonstrates the basic structure of such an xml file: The root tag is a \textit{Map} element. Within this map, you can define several sections indicated with flags (e.g. \textbf{A} or \textbf{B}). Later, in the code, you will specify one flag while loading the level. The engine will than load every game object defined within the corresponding section. Maybe you have noticed the third section having quite a weird flags attribute: \textbf{$A|B$}. You can use this \textbf{$|$} character, to define sections that are loaded if either the left or the right side matches the loading flag. Thus \textbf{$A|B$} is loaded whenever you load with flag A or with flag B. Sure, you can combine even more flags: \textbf{$A|B|C|D$}. We use the flag attribute to differentiate between objects only created on a server and others created only on clients, but you can use them for any other purpose. And last but not least, each Attribute tag defines one attribute you would like to pass to the template for which you used the string map earlier. But that was kinda obvious, I guess.

\lstinputlisting[language=XML, style=basic, caption={level.xml}, label=code:xml]{listings/level.xml}

\section{Loading Levels}

We're nearly done. The xml is written. Now comes the C++ code to load a level.

\begin{lstlisting}[language=C++, style=basic, caption={loading a level}, label=code:load]
i6engine::api::EngineController::getObjectFacade()->loadLevel("level.xml", "A");
\end{lstlisting}

Yeah, you're done. Pass the path to the level as well as the flag specifying the section(s) you want to load and the rest is done by the engine.

\section{Terrain}

You may have noticed the new GameObject \textit{map} used in the xml file. A Terrain is one of the very few GameObject templates offered with the engine. You can use them just like your own ones. For a terrain, you have to specify the heightmap in \textit{map}, the texture layers in \textit{layers}, the size extends in \textit{size}, the height scale factor in \textit{inputScale}, the different terrain segments with \textit{minX}, \textit{maxX}, \textit{minY} and \textit{maxY} and the textures for diffuse and specular map in \textit{layer\_{}0\_{}diffusespecular} and for displacements and normal map in \textit{layer\_{}0\_{}normal}. Everything else is done by the engine. Currently, only flat terrains are supported completely (no height collision shape), but this will hopefully change in the near future.

\section{Conclusion}

You can define your level within an xml file. Thus you can easily switch between levels ingame and you can modify your level without recompiling your game.

\section{What's next?}

The next tutorial will show you how to use physic.

\end{document}
